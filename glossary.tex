
% From https://www.overleaf.com/learn/latex/Glossaries

\makeglossaries % Prepare for adding glossary entries


%\newglossaryentry{latex}
%{
%        name=latex,
%        description={Is a mark up language specially suited for
%scientific documents}
%}

%\newglossaryentry{bibliography}
%{
%        name=bibliography,
%        plural=bibliographies,
%        description={A list of the books referred to in a scholarly work,
%typically printed as an appendix}
%}

\newglossaryentry{Theia}
{
  name=Theia,
  description={An \acrshort{IDE} for software development. Theia is accessible in a web browser and as a desktop application. The implementation reuses much of \gls{VSCode}'s internals. Managed by the Eclipse Foundation}
}

\newglossaryentry{VSCode}{
  name=VSCode,
  description={Visual Studio Code. An \acrshort{IDE} for software development. The full name is Visual Studio Code. Managed by Microsoft}
}

\newglossaryentry{metamodel}{
  name=metamodel,
  description={In conceptual modeling, the model itself can be modeled. This model of the model is called a meta-model. \Gls{Ecore} is one example of a metamodel}
}

\newglossaryentry{Ecore}{
  name=Ecore,
  description={The \acrshort{emf} core model. A \gls{metamodel} similar to UML Class Diagrams}
}

\newglossaryentry{Eclipse}{
  name={Eclipse IDE},
  description={An \acrshort{IDE} by the Eclipse Foundation. Originally created by IBM. It is based on a plugin architecture using OSGi, and is written in Java. For more details, see \cref{sec:eclipse-ide}}
}

\newglossaryentry{cloud}{
  name={cloud},
  description={Remote data centers that provide computing as a service. Commonly used by businesses to provide web infrastructure}
}

\newglossaryentry{open source}{
  name={open source},
  description={The source code for a software is available; not just for inspection, but for re-use and modification}
}

\newglossaryentry{Che}{
  name={Eclipse Che},
  description={A \gls{cloud} based or self hosted workspace and \acrshort{IDE} for software development. It is based on \gls{Kubernetes} and \gls{Theia}}
}

\newglossaryentry{Gitpod}{
  name=Gitpod,
  description={A \gls{cloud} based workspace and \acrshort{IDE} for software development. It is based on \gls{Docker}, \gls{Kubernetes} and \gls{Theia}}
}

\newglossaryentry{Kubernetes}{
  name=Kubernetes,
  description={A container orchestration system created by Google}
}

\newglossaryentry{Docker}{
  name=Docker,
  description={A linux containerization system for process isolation and distribution. Popular for distributing and deploying self-contained software applications}
}

\newglossaryentry{UML}{
  name=UML,
  description={Unified Modeling Language. A common modeling language for creating diagrams such as Class Diagrams. It is standardized by \acrlong{OMG}}
}

\newglossaryentry{Electron}{
  name={Electron},
  description={A desktop application runtime for javascript, based on Chromium}
}
\newglossaryentry{Nodejs}{
  name={NodeJS},
  description={A javascript interpreter for desktop, based on the Chromium V8 javascript engine. It also includes some desktop \acrshortpl{API} like filesystem access}
}

\newglossaryentry{JSON-RPC}{
  name={JSON-RPC},
  description={Remote Procedure Call (RPC) protocol using Javascript Object Notation (JSON) serialization. It allows a process to execute functions in another process and obtain the results}
}

\newglossaryentry{JSON}{
  name={JSON},
  description={Javascript Object Notation. A serialization format for object structures}
}

\newglossaryentry{REST}{
  name=REST,
  description={Represential State Transfer (REST). A paradigm for creating HTTP \acrshortpl{API}, centered around resources}
}

\newglossaryentry{WebSocket}{
  name=WebSocket,
  description={A two-way communication protocol over TCP sockets made available for web browsers. It allows for a persistent and reusable connection which can send multiple messages, unlike regular HTTP requests. Commonly used to avoid polling over HTTP, or live updates of a website}
}

\newglossaryentry{NPM}{
  name={NPM},
  description={Node Package Manager (NPM). Hosts javascript packages online. Similar to Maven Central, but for javascript}
}

\newglossaryentry{ELK}{
  name={Eclipse Layout Kernel},
  description={Eclipse Layout Kernel (ELK) is a system for automatic diagram layout.~\cite{eclipsefoundationEclipseLayoutKernel} It positions nodes and edges for optimal viewing}
}

%\newglossaryentry{API}{
%  name=API,
%  description={Application Programming Interface (API). The interface which  another program needs to use. This can be protocols and message contents, or  programming language constructs like function definitions}
%  long={Application Programming Interface},
%  first={\glsentrylong{api}} (\glsentryname{api})
%}

% --------------------
% ----- Acronyms -----
% --------------------

%\newacronym{phd}{PhD}{philosophiae doctor}
%\newacronym{CoPCSE}{CoPCSE@NTNU}{Community of Practice in Computer ScienceEducation at NTNU}
%\newacronym{gcd}{GCD}{Greatest Common Divisor}

\newacronym{mdd}{MDD}{Model-Driven Development}
\newacronym{IDE}{IDE}{Integrated Development Environment}
\newacronym{emf}{EMF}{Eclipse Modeling Framework}
\newacronym{NTNU}{NTNU}{Norges Teknisk-naturvitenskapelige Universitet}
\newacronym{API}{API}{Application Programming Interface}

\newacronym{EMOF}{EMOF}{Essential Meta Object Facility}
\newacronym{OMG}{OMG}{Object Management Group}
\newacronym{XMI}{XMI}{XML Metadata Interchange}

\newacronym{RCP}{RCP}{Rich Client Platform}
\newacronym{RAP}{RAP}{Remote Application Platform}
\newacronym{GWT}{GWT}{Google Web Toolkit}

\newacronym{LSP}{LSP}{Language Server Protocol}
\newacronym{GLSP}{GLSP}{Graphical Language Server Platform}
