
% From https://www.overleaf.com/learn/latex/Glossaries

\makeglossaries % Prepare for adding glossary entries


%\newglossaryentry{latex}
%{
%        name=latex,
%        description={Is a mark up language specially suited for
%scientific documents}
%}

%\newglossaryentry{bibliography}
%{
%        name=bibliography,
%        plural=bibliographies,
%        description={A list of the books referred to in a scholarly work,
%typically printed as an appendix}
%}

\newglossaryentry{Theia}
{
  name=Theia,
  description={An \acrshort{IDE} for software development. Theia is accessible in a web browser and as a desktop application. The implementation reuses much of \gls{VSCode}'s internals. Managed by the Eclipse Foundation}
}

\newglossaryentry{VSCode}{
  name=VSCode,
  description={Visual Studio Code. An \acrshort{IDE} for software development. The full name is Visual Studio Code. Managed by Microsoft}
}

\newglossaryentry{metamodel}{
  name=metamodel,
  description={In conceptual modeling, the model itself can be modeled. This model of the model is called a meta-model. \Gls{Ecore} is one example of a metamodel}
}

\newglossaryentry{Ecore}{
  name=Ecore,
  description={The \acrshort{emf} core model. A \gls{metamodel} similar to UML Class Diagrams}
}

\newglossaryentry{Eclipse}{
  name={Eclipse IDE},
  description={An \acrshort{IDE} by the Eclipse Foundation. Originally created by IBM. It is based on a plugin architecture using OSGi, and is written in Java}
}

\newglossaryentry{cloud}{
  name={cloud},
  description={Remote data centers that provide computing as a service. Commonly used by businesses to provide web infrastructure}
}

\newglossaryentry{open source}{
  name={open source},
  description={The source code for a software is available; not just for inspection, but for re-use and modification}
}

\newglossaryentry{Gitpod}{
  name=Gitpod,
  description={A \gls{cloud} based workspace and \acrshort{IDE} for software development. It is based on \gls{Docker}, \gls{Kubernetes} and \gls{Theia}}
}

\newglossaryentry{Kubernetes}{
  name=Kubernetes,
  description={A container orchestration system created by Google}
}

\newglossaryentry{Docker}{
  name=Docker,
  description={A linux containerization system for process isolation and distribution. Popular for distributing and deploying self-contained software applications}
}

%\newglossaryentry{API}{
%  name=API,
%  description={Application Programming Interface (API). The interface which  another program needs to use. This can be protocols and message contents, or  programming language constructs like function definitions}
%  long={Application Programming Interface},
%  first={\glsentrylong{api}} (\glsentryname{api})
%}

% --------------------
% ----- Acronyms -----
% --------------------

%\newacronym{phd}{PhD}{philosophiae doctor}
%\newacronym{CoPCSE}{CoPCSE@NTNU}{Community of Practice in Computer ScienceEducation at NTNU}
%\newacronym{gcd}{GCD}{Greatest Common Divisor}

\newacronym{mdd}{MDD}{Model-Driven Development}
\newacronym{IDE}{IDE}{Integrated Development Environment}
\newacronym{emf}{EMF}{Eclipse Modeling Framework}
\newacronym{NTNU}{NTNU}{Norges Teknisk-naturvitenskapelige Universitet}
\newacronym{API}{API}{Application Programming Interface}
