\chapter{Reviewed Research Questions}\label{chap:review-research-questions}

The research questions were presented in \cref{chap:research-questions}.
Some of them can be answered, at least to some extent, by the background material in \cref{chap:background}.

\section{The Old Research Questions}

\paragraph*{Modernizing \acrshort{mdd}}
First, lets look at \cref{rq:1}:
\begin{displayquote}
  How can we modernize \acrlong{mdd} Frameworks to appeal to the next generation of software developers, using recent developments in \gls{cloud} \acrshortpl{IDE}?
\end{displayquote}

It appears this modernization is already in progess.
With the introduction of \emph{Sirius Web} (\cref{sec:sirius-web}) and \emph{ecore-glsp} (\cref{sec:ecore-glsp}), \gls{Ecore} editors in \gls{cloud} based \glspl{IDE} are coming.
In addition, \gls{Ecore} code generation can already target web browsers using \emph{ecore.js} or \emph{crossecore-typescript}. % TODO: link to chapters


The use of more a more popular serialization format, \gls{JSON} instead of XML/\gls{XMI}, is being used to \textit{some} extent.
Most of the current efforts are still serializing to disk as \gls{XMI}, using only \gls{JSON} inside protocols for data transfer.
This \emph{could} be an advantage, however, to keep the interoperability between different tools.
Most of the developers' work is not done directly in these files, but instead with graphical interfaces. Therefore, this is perhaps not a big issue.


The biggest blocker right now is just implementation: not all features are supported yet.
Another concern is targeting only \gls{Theia}, as is the case for ecore-glsp with its use of Theia Extension (\cref{sec:theia-extension}).
This prevents usage in \gls{VSCode}\footnote{Although not a \gls{cloud} editor (yet? \href{https://github.com/features/codespaces}{https://github.com/features/codespaces}), it is very popular.~\cite{stackoverflowStackOverflowDeveloper2019}}, and also in Gitpod.
It could possibly be used in Che, by customizing the \gls{IDE} used\footnote{This may need help from a system administrator. Alternatively, a user can compile their own Docker image with a modified Theia instance, and specify it in a devfile.~\cite{maxshaposhnikIntroductionDevfile2020}}.


\paragraph*{Essential tools}
Next up is \cref{rq:2}:
\begin{displayquote}
  What tools are essential to use \acrlong{mdd} in the \gls{cloud}?
\end{displayquote}

The model editor seems like the obvious answer. 
Two different types of model editor were found.
There are tree and property editors, and graphical diagram editors.
These should support both the general \gls{Ecore} model and model instances.
Some editors also support the genmodel.

In addition, there is validation, comparison and code generation.
Of these, code generation seems most essential, and validation second.
The ongoing development in \cref{sec:ecore-editors} on ecore-glsp also seems to share this view: diagram editor first, then code generation, then validation last\footnote{Ecore-glsp already supports genmodel and EMF Codegen: \href{https://github.com/eclipse-emfcloud/ecore-glsp/issues/10}{https://github.com/eclipse-emfcloud/ecore-glsp/issues/10}.}.

Even though personal experience with \gls{emf} relied mainly on a tree editor, little focus has been on implementing \gls{Ecore} tree editors in the \gls{cloud} yet.
There is the generic Theia Tree Editor (\cref{sec:theia-tree-editor}), but it has not been applied to \gls{Ecore} yet.
A generic tree editor could work with models, model instances and the genmodel, making it a ``one solution to rule them all'', compared to diagram editors.
This is what \gls{Eclipse} has already, like the Sample Reflective Ecore Model Editor and the EMF Forms based editor.

Validation and code generation tools could be embedded in VSCode extensions as executables, and ran as VSCode commands or from a tree editor.
They do not require special user interfaces.

\paragraph*{A good tool}
Perhaps more on the side of \textit{usability}, or \textit{minimum viable product} requirements, is \cref{rq:3}:
\begin{displayquote}
  What does a good \gls{cloud} based tool for \acrlong{mdd} look like?
\end{displayquote}

What it \emph{looks like} is not mainly about visuals, but about its software architecture and functional requirements.
\textit{What does it do}?
And \textit{how} does the tool do something?

First off, flexibility seems to be key.
Flexibility and extensibility in the protocols used, and configurability with regards to details.
It is hard to define everything up front, so some leeway for tool developers is left in.
Custom protocol messages; optional functionality.
Using dependency injection in order to create an extendible system seems popular in this cohort of developers as well.
For example \gls{Theia}, Sprotty and \gls{GLSP} all do this.

Second is configuration. 
It is hard to predict everything, so exposing details as configuration options allows solutions to arise for unknown situations.
An example is \textit{java}: a lot of the \gls{emf} ecosystem exists as java code, and could be embedded as \texttt{.jar} files, but \emph{where is java installed} on the workspace operating system, and which java version?
With new environments like Gitpod and Che, some old assumptions do not hold\footnote{Example: you assume \texttt{java} can be invoked from the OS Path, but what if it is ran as a Docker container instead with \texttt{docker run}?}.

Third is to augment models instead of extending their serialization format.
Most of the tools store some extra data, like layout or options.
These pieces of information are stored in files separate from the \texttt{.ecore} model, such as \texttt{.aird}, \texttt{.json} and \texttt{.genmodel}.
Common is the usage of \gls{XMI}, but some use \gls{JSON}.

Fourth is a client-server separation, both because \gls{Theia} is distributed, but also to have reusable (generic) clients and reusable servers like \gls{LSP}.
Runtime dependencies may dictate how code is organized, for extensions that work across browsers, \gls{Electron} and \gls{Nodejs}.

% Fifth: using Ecore models internally? Used by some: Command API, Tree structure, maybe more

% Sixth: look for requirements in existing tools, like validation, undo/redo, refacotring, references, renaming etc?

\paragraph*{Reusing tools and architectures}
The last question is \cref{rq:4}:
\begin{displayquote}
  How can existing modeling tools and software architectures be reused in the new \gls{cloud} based modeling tool?
\end{displayquote}

It seems existing tools that can be ran standalone as servers or command line tools can be reused.
For example, \gls{LSP} shows this, by letting language servers execute what they want internally, like existing compilers.
Bridging the gap between old programs and new could be done by exposing them over an \gls{API}, such as \gls{REST} or \gls{JSON-RPC}. The Model Server (\cref{sec:model-server}) does this, and hides away java and \gls{Ecore} model parsing behind a \gls{REST} \acrshort{API}.

% TODO: write some more. Architecture. LSP, Model server, generic frontends.

\section{New Research Questions}

\begin{questions}[leftmargin=1cm]
  \item \emph{How can we design an Ecore master-detail tree editor that works in both VSCode and Theia, while reusing existing tools for Ecore such as codegen and validation?}\label{rq:21}
  \begin{questions}
    \item \emph{How can java applications be used in a VSCode extension?}\label{rq:22}
    \item \emph{How should the editor cooperate with multiple other tools that change the same underlying model?}\label{rq:23}
    \item \emph{What data is required for displaying a model as a tree?}\label{rq:24}
    \item \emph{How can a tree editor component be generic enough to support arbitrary user actions?}\label{rq:25}
    \item \emph{How can the user interface know the model well enough so the user is constrained from creating invalid Ecore trees?}\label{rq:26}
  \end{questions}
\end{questions}
