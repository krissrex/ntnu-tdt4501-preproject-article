\chapter*{Sammendrag}

Model-dreven utvikling (MDU) er en tilnerming innen programvareutvikling.
Et vanlig frammeverk for MDU er Eclipse Modeling Framework.
Desverre gir studenter motstand når de skal lære dette rammeverket fordi den bruker upopulære teknologier.
Denne avhandlingen kommer før en masteroppgave, og ser på hvordan modelering kan flyttes til skybaserte redigeringsprogram for å fornye teknologien som brukes.
Samtidig ser den etter muligheter for gjenbruk av eksisterende implementasjoner, arkitekturer og protokoller for å spare arbeidsinnsats.
Avhandlingen slutter med å finne et behov for et tre-basert redigeringsverktøy, og tester hvorvidt det er gjennomførbart.
Til slutt presenteres funksjonelle krav og en arkitektur for et slikt redigeringsverktøy.
Disse kravene kan ses på videre og implementeres i en masteroppgave. 
