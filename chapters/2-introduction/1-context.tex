\section{Context}

%Software engineering + modeling
\paragraph*{Software Engineering and Modeling}
Software engineering is a broad field. It has multiple approaches to methodology, programming languages, practices and paradigms.
One approach to developing software systems is to model a domain.
The domain is usually a phenomena in the real world which a business is interested in. And the conceptual model is an abstraction of this domain, where essential actors, entities, properties and relationships are formalized.

% MDSE, MDD, MBD
\paragraph*{Model usage}
This conceptual model can be utilized by the engineers to various degrees.
It could simply document the domain in order to help understanding and communication among those working with the domain. Or the model could be the basis of a software system. The software could be generated automatically based on the model, or an interpreting system could execute the model itself.

% MDD
\paragraph*{Model-Driven Development}
When the model itself is the source of truth for software, the model is \textit{driving} the development. This is called \acrfull{mdd}. Changes to the domain result in changes to the model, before changing the system's code. And the code itself is \textit{derived from the model}, often using model-to-text transformations. Oftentimes a \textit{complete} system can not be generated, and a programmer is required to manually code some details or rules of the domain into the generated code.

% Education of MDD
\paragraph*{Education and MDD}
The concepts and tools used in \acrshort{mdd} are being taught to software engineering students.
Not all curriculums include \acrshort{mdd}, as it is mostly relevant for specializations in Software Development. %Skip sentence?
Development using models is very reliant on tools. One of the tools that students learn is the \acrfull{emf} and its \gls{metamodel} called \gls{Ecore}.
An advantage of \acrshort{emf} is the \gls{open source} nature, which allows students to look inside its implementation, and use it for free.\footnote{Compare this to Matlab+Simulink, which some students learn. It costs 20000 NOK per year for a license.}

% EMF details
\paragraph*{EMF in detail}
The \acrlong{emf} allows a modeller to create a representation of their domain, and generate java code. The java code has classes for all the modeled entities.
It can also generate an \textit{editor}, which is a user facing application.
This editor lets a user enter objects from their domain. For example, if the domain was a book store, the model would be \textit{books} with \textit{names} and \textit{authors}. And an object could be \textit{``Harry Potter''} by \textit{``J. K. Rowling''}. The generated editor uses \gls{Eclipse} as its platform, extending it with an \textit{Eclipse plugin}.

% Issues with EMF
\paragraph*{Issues with EMF}
Despite \acrshort{emf} being a great framework, its tight integration with \gls{Eclipse} harms the adoption. Some students have a negative attitude against \gls{Eclipse}, and learning about its architecture feels like wasted time.
\textcite{kuzniarzTeachingModelDrivenSoftware2016} found that students resist when the technology and skills are not used in other courses.
A general problem for \acrshort{mdd} adoption identified by \textcite{jonwhittleTaxonomyToolrelatedIssues2015} was a low impact on personal career needs. 
The students look to the industry, and mostly find mentions of ``Cloud, IntelliJ, React, Typescript, Scrum, AI'', never Eclipse. They forget that much of the popular technologies are hype driven, and not the only tools used.
\textcite[p.~21--23]{brambillaModelDrivenSoftwareEngineering2017} claims that model-driven software engineering has passed the peak of the hype curve, and is now in the ``slope of enlightenment''. This means \acrshort{mdd} is not dead, but rather that we are only now seeing how it is best used.\footnote{Compare this to how the hyped Blockchain was supposed to fix every problem, but now we see it doesn't solve much at all.} A co-author of \cite{brambillaModelDrivenSoftwareEngineering2017} tried teaching students in 2015, and collected their feedback. Much of the complaints were about installation issues and problems with the tools. \cite{jordicabotFailedConvinceMy2015}

% My proposal
\paragraph*{The proposed solution}
A platform shift from the \gls{Eclipse} desktop application to the \gls{cloud} could alleviate many of the problems. An editor in the \gls{cloud} would have to be web-based, and could use many of today's modern (and popular) technologies. 
Another win is that an ``editor-as-a-service'' does not need installation.
As mentioned, this caused many problems in \cite{jordicabotFailedConvinceMy2015}. 
Editors based on web technology is not unheard of either. One such editor that has become very popular, is Microsoft's \gls{VSCode}.\footnote{This thesis is written using \gls{VSCode}!}
In fact, the Eclipse Foundation itself has picked up on this trend, and accepted a web-based editor called \gls{Theia} into their umbrella of projects.
This means that the platforms to build a \gls{cloud} based modeling editor are here. 
This is not a new thought either, as \textcite{lajmiModelingBrowserWhat2016} said in 2016: \textquote{the next wave of modeling tools will probably be based on cloud IDEs.}
On the other hand, recreating the entire \acrlong{emf} is no easy task. Therefore, its tooling for validation, serialization etc.\ should be reused if possible.


% Related works and their shortcomings
\paragraph*{Related works}
% 1,2,9,10
% 1,2 existing model
% 2,9 textual
% 10 not ecore
% 9 too early, before sprotty

% 1: carrascal-manzanaresBuildingMDECloud2015
% 2: coulonModularDistributedIDE2020
% 9: sainiWebCollaborativeModelling2019
% 10: walshClientagnosticHybridModel2020

Some research related to \gls{mdd} and the \gls{cloud} has been done already.
In \cite{carrascal-manzanaresBuildingMDECloud2015} they discuss using an existing model in the cloud.
\textcite{coulonModularDistributedIDE2020} also uses an existing model for their deployment to a \gls{cloud} \acrshort{IDE}.
\textcite{sainiWebCollaborativeModelling2019} discusses modeling in the web, but at an early stage before tools like Sprotty (\cref{sec:sprotty}) were ready for use.
In \cite{walshClientagnosticHybridModel2020} they create a model editor, but for a language called \texttt{UML-RT}.

None of these earlier works look at modeling in the cloud where the goal is to create \gls{Ecore} models. They either \emph{deploy} \gls{Ecore} models (\cite{carrascal-manzanaresBuildingMDECloud2015,coulonModularDistributedIDE2020}), are textual editors (\cite{coulonModularDistributedIDE2020,sainiWebCollaborativeModelling2019}) or not modeling with \gls{Ecore} (\cite{walshClientagnosticHybridModel2020}).