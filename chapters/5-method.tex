\chapter{Method}

\paragraph*{How to answer the sub-research questions}
To answer the sub-research questions \cref{rq:22} to \cref{rq:26}, a design and creation approach will be used.
Developing prototypes and testing them in the real environment can quickly remove doubts about feasibility, as long as the prototype works.
(In case a prototype fails to demonstrate a working approach, a new approach will have to be designed and tested in a new prototype.
If no new design can be made, then the background material needs further research.)

\paragraph*{How to answer the main research question}
To answer the main question, \cref{rq:21}, an analysis of existing tools will form the requirements for the editor.
The findings from \cref{rq:22} to \cref{rq:26} will influence the solution, and existing architectures and protocols will also be used as inspiration.

The answer to \cref{rq:21} will be a design, like diagrams and requirements.
It will not be a full implemented software solution, because of the limited scope of this pre-master's thesis specialization project.
An implementation will instead be explored in the following master's thesis.

\paragraph*{Data generation method}
To find data that can answer the research questions, I will use documentation on the internet from \gls{open source} software related to modeling.
Discovering the related software and documentation will be done by:
\begin{itemize}
  \item exploring the Eclipse Foundation's project umbrellas for related software
  \item inspecting the source code of existing \gls{Eclipse} \gls{Ecore} editors (discussed in \cref{sec:ecore-editors})
  \item searching with internet search engines (Google, GitHub) for \gls{emf} related libraries
  \item viewing presentations from \emph{EclipseCon 2020}\footnote{This is a developer conference by Eclipse Foundation. The 2020 conference has an agenda that aligns very well with modeling in the \gls{cloud}.} about \gls{emf} modeling, Sirius Web (\cref{sec:sirius-web}), \acrshort{GLSP}
  \item ask questions during\footnote{The conference happened during this thesis, and was virtual, so this was possible.} or after EclipseCon to presenters or key personalities 
  \item inspecting GitHub organizations that contribute modeling projects, to see if they offer other related projects
  \item inspecting developer documentation, guides and wiki pages for software, to find related documents
  \item inspect software dependencies\footnote{Usually listed in files like \texttt{pom.xml} and \texttt{package.json}} in discovered libraries, to find other related libraries 
\end{itemize}

\paragraph*{Data analysis}
The results will be analyzed qualitative to give a clear view of the state of the art, current developments, possibilities for new development and insight into the architectures and protocols already in use.
Together, this will form the basis of a new tool's requirements and design.



% What is a good solution
%  - How it fits into an existing ecosystem

% TODO: write

% Software requirements for the results

% Solve RQ2.1 with prototype 1
% Solve RQ2.2 with model server
% Solve RQ2.3 with prototype 2
% Solve RQ2.4 with prototype 2
% Solve RQ2.5 with prototype 2 [WIP]

% Answer RQ2 with diagrams from design-doc of prototype 2?
