\section{Protocols}\label{sec:protocols}

This section will describe some of the protocols used by the editors in \cref{sec:editors} and extensions in \cref{sec:extension-components}.

\subsection{Language Server Protocol (LSP)}\label{sec:lsp}

\paragraph*{Goal} An \gls{IDE} often has to support many programming languages.
Most of the languages support some common features, such as autocomplete,  validation, definitions, references, renaming, selection etc..
The \gls{LSP} tries to separate language editor clients from language servers.
The clients are kept generic, while the language servers know the details for a programming language.~\cite{microsoftOverview}
This means a \gls{IDE} developer only needs to create \textit{one} editor, with generic \gls{LSP} support.
And a language developer only needs to create a language server.
If the language server adheres to the \gls{LSP} protocol, any \gls{LSP} client will automatically support it.
This is illustrated in \cref{fig:lsp-benefits}.

\begin{figure}[htbp]  % order of priority: h here, t top, b bottom, p page
  \centering
  \includegraphics[width=\textwidth]{figures/lsp-languages-editors}
  \caption[LSP Benefits]{The benefits of using \gls{LSP}. There is no need for re-creating language support for every possible combination of editor and language.~\cite{microsoftLanguageServerExtension2020}}\label{fig:lsp-benefits}
\end{figure}


\paragraph*{Details}
The protocol is based on a \emph{Base protocol}, which has a \texttt{key-value} \emph{header} and a \emph{content} with \gls{JSON-RPC} (see \cref{fig:lsp-protocol}).

\begin{figure}[htbp]
  \centering
  \includegraphics[width=\textwidth]{figures/lsp-protocol}
  \caption[The LSP Protocol]{The LSP protocol extends a Base protocol that uses HTTP headers and JSON-RPC content.}\label{fig:lsp-protocol}
\end{figure}

\paragraph*{The Base protocol}
The header fields conform to the HTTP specification\footnote{RFC-7230 at \href{https://tools.ietf.org/html/rfc7230\#section-3.2}{https://tools.ietf.org/html/rfc7230\#section-3.2}.} with regards to structure and formatting.

The Base protocol defines tree different message types that are sent in the \gls{JSON-RPC} content: \emph{Request}, \emph{Response} and \emph{Notificaion}.
There are also two key rules: every Request must be answered with a Response, and Notificaion does not need a Response.
The Base protocol provides a list of common error codes, extending the default error codes that \gls{JSON-RPC} provides.~\cite{microsoftLanguageServerProtocol2020}

The Base protocol defines a rule for Request and Notification methods as well: a method starting with \texttt{\$/} is protocol implementation specific and therefore optional, meaning a client or server can skip implementing it if the method is not suitable\footnote{The example given is cancellation of asynchronous tasks in a server that is synchronous.}.~\cite{microsoftLanguageServerProtocol2020}

Lastly, the Base protocol defines two Notifications: \texttt{\$/progress} and \texttt{\$/cancelRequest}.~\cite{microsoftLanguageServerProtocol2020}

\paragraph*{The Language Server Protocol}
The actual \emph{Language Server Protocol} itself is a large collection of \gls{JSON-RPC} Request, Response and Notification messages sent over the Base protocol.
The protocol assumes that \textit{one} language server is used for \textit{one} language client.~\cite{microsoftLanguageServerProtocol2020}
This means a language server can not be shared between multiple clients.

The protocol is tailored for textual languages and text documents.
It specifies a collection of data structures that should be used.
A central data structure is the \texttt{TextDocument}.
Inside a \texttt{TextDocument}, there can be \texttt{Location}s in the text, and \texttt{Range}s of text between two \texttt{Location}s.
Changing a \texttt{TextDocument} is done by constructing a \texttt{TextDocumentEdit}, which holds a list of text replacements in \texttt{TextEdit}s, and a pointer to a \texttt{TextDocument} version using a \texttt{VersionedTextDocumentIdentifier}.
A \texttt{TextEdit} has \texttt{Range} to edit and a string with the new text.~\cite{microsoftLanguageServerProtocol2020}
An illustration of the data structures are shown in \cref{fig:lsp-data-structures}.
Note that this is only a small subset of the data structures defined in \gls{LSP}\footnote{Most of the supported features are described at \href{https://code.visualstudio.com/api/language-extensions/language-server-extension-guide\#additional-language-server-features}{https://code.visualstudio.com/api/language-extensions/language-server-extension-guide\#additional-language-server-features}.}.

\begin{figure}[htbp]
  \centering
  \includegraphics[width=\textwidth]{figures/lsp-textdocument-data}
  \caption[LSP Central Data Structures]{Central data structures in \gls{LSP}.}\label{fig:lsp-data-structures}
\end{figure}


\subsection{Graphical Language Server Platform (GLSP)}\label{sec:glsp}

% TODO: write

\subsection{JSON-RPC}
% TODO: write