\subsection{Prototype 2}

This prototype tries to answer \cref{rq:23}, \cref{rq:24}, \cref{rq:25} and \cref{rq:26}:
\begin{displayquote}
  How should the editor cooperate with multiple other tools that change the same underlying model?\\
  What data is required for displaying a model as a tree?\\
  How can a tree editor component be generic enough to support arbitrary user actions?\\
  How can the user interface know the model well enough so the user is constrained from creating invalid Ecore trees?
\end{displayquote}

\subsubsection{Requirements}

Cooperation in \cref{rq:23} could possibly be solved by channeling and model edits through a Model Server. An example of this was seen in the \emph{Coffee Editor} reference implementation for \gls{GLSP}\footnote{}.



\begin{itemize}
  \item Develop using VSCode extension \acrshort{API} (see \cref{sec:vscode-extension}). No dependencies to Theia.
  \item Visualize a tree structure in the editor, based on a object structure (e.g. \gls{JSON}).
  \item Show labels and icons for nodes in the tree. Labels will be from node \emph{type}, labels from node data.
  \item Add and remove nodes in the tree from user interaction.
  \item Property sheet editor that is synchronized with tree selection.
  \item Visualize both an Ecore meta-model and model instance.
  \item Use EMF.Cloud Model Server (\cref{sec:model-server}) to get the model. Do not read the \texttt{.ecore} file from disk in the extension.
  \item Automatically update the views when the underlying model changes (push not poll)
\end{itemize}

The \textit{design document} used for implementation is added in \cref{app:prototype-2-design-doc}.

\subsubsection{Implementation}
%TODO: write

\subsubsection{Results}
%TODO: write

