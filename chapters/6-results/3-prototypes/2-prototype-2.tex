\subsection{Prototype 2}

This prototype tries to answer \cref{rq:23}, \cref{rq:24}, \cref{rq:25} and \cref{rq:26}:
\begin{displayquote}
  How should the editor cooperate with multiple other tools that change the same underlying model?\\
  What data is required for displaying a model as a tree?\\
  How can a tree editor component be generic enough to support arbitrary user actions?\\
  How can the user interface know the model well enough so the user is constrained from creating invalid Ecore trees?
\end{displayquote}

\subsubsection{Requirements}

\paragraph*{Cooperation (\cref{rq:23})}
Cooperation in \cref{rq:23} could possibly be solved by channeling any model edits through a Model Server, and listening to this server for change notifications. An example of this was seen in \cref{sec:coffee-ide} with the \emph{Coffee Editor} reference implementation for \gls{GLSP} (\cref{sfig:coffee-example-glsp}).

\paragraph*{Tree model (\cref{rq:24})}
By displaying an example tree, the required data will be evident as the view is implemented.

\paragraph*{Generic actions (\cref{rq:25})}
By supporting actions through buttons in the user interface, but not hardcoding their clicks to trigger an action, they can be generic.
This means a click must only notify something that an \emph{intent} to trigger action \texttt{XYZ} has happened.
Implementing a skeleton for this would show what data is required for generic actions.

\paragraph*{Constrain tree for validity (\cref{rq:26})}
This is relevant especially for drag-and-drop, where a user should not be able to drop a node onto an ``incorrect'' parent.
Due to time constraints, drag-and-drop will not be implemented now.
But the supporting data structure could be sketched out, based on the previous data structures for the Tree model in \cref{rq:24}.

\paragraph*{Here is the list of identified requirements for this prototype:}

\begin{itemize}
  \item Develop using VSCode extension \acrshort{API} (see \cref{sec:vscode-extension}). No dependencies to Theia.
  \item Visualize a tree structure in the editor, based on a object structure (e.g. \gls{JSON}).
  \item Show labels and icons for nodes in the tree. Labels will be from node \emph{type}, labels from node data.
  \item Add and remove nodes in the tree from user interaction.
  \item Property sheet editor that is synchronized with tree selection.
  \item Visualize both an Ecore meta-model and model instance.
  \item Use EMF.Cloud Model Server (\cref{sec:model-server}) to get the model. Do not read the \texttt{.ecore} file from disk in the extension.
  \item Automatically update the views when the underlying model changes (push not poll)
\end{itemize}

The \textit{design document} used for implementation is added in \cref{app:prototype-2-design-doc}.

\subsubsection{Implementation}
%TODO: write

\subsubsection{Results}
%TODO: write

